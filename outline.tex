\documentclass[12pt]{article}
\raggedright
\addtolength{\textheight}{2 in}
\addtolength{\oddsidemargin}{-.5in}
\addtolength{\textwidth}{1.0 in}
\addtolength{\topmargin}{-1in}
\parindent0in
\usepackage{color}
\usepackage{url}
\newcommand{\ben}{\begin{enumerate}}
\newcommand{\een}{\end{enumerate}}
\usepackage{multicol}
\begin{document}

 \centerline{\bf Physics 718\hspace{3mm} White dwarfs, neutron stars, black holes, and related topics\hspace{3mm} Spring 2019} 

\ben
\item[I.]  Preliminaries
\begin{enumerate}
\item[I.1] Precision
\item[I.2] Small Angles
\item[I.3] Units and Celestial Constants 
\end{enumerate}

\item[II.] Spherical Newtonian Stars
\begin{enumerate}
\item[II.1] Stellar radiation
\item[II.2] Stellar Structure: Hydrostatic Equilibrium of a Newtonian Star
\item[II.3] Hydrodynamics: ${\bf F} = m{\bf a}$ for a fluid element
\item[II.4] Properties of Stellar Equilibria and the Virial Theorem
\item[II.5] The Euler equation 
\end{enumerate}

\item[III.] Mathematical prerequisites
\ben
\item[III.1] Lie derivatives; exterior derivatives
\item[III.2] Dragging along and pulling back. Tensors on submanifolds
\item[III.3] Integration on manifolds: Stokes' theorem
\een

\item[IV.]  Stellar structure
\ben
\item [IV.1] Euler equations and equilibrium equations
\item [IV.2] Spherical stellar models. Lane-Emden equation
\item [IV.3] Relativistic equations of a perfect fluid
\item [IV.4] Spherical relativistic models: the TOV equations
\item [IV.5] Simplest relativistic models --- uniform density stars and polytropes
\een

\item[V.] Structure of white dwarfs and the cold equation of stat below neutron drip (S\&T Chapters 2-3) 
\ben
\item[V.1]  Elementary estimates 
\item[V.2]  Relativistic thermodynamics
\item[V.3]  Ideal fermi gas
\item[V.4]  White dwarf cooling and observation ? 
\item[V.4]  HW and BPS equations of state
\item[V.5]  Mass-radius relation and Chandrasekhar limit
\item[V.6]  Collapse and Supernovae
\een

\item[VI.]  Neutron stars and the equation of state above neutron drip
\ben
\item[VI.1] Baym-Bethe-Pethick EOS
\item[VI.2] More sophisticated nuclear EOSs
\item[VI.3] Spherical stellar models
\item[VI.4] Observational constraints on the equation of state
\een

\item[VII.] Stellar stability
\ben
\item[VII.1]  Convective instability
\item[VII.2]  Newtonian instability to collapse
\item[VII.3]  Relativistic instability to collapse and turning points.
\een

\item[VIII.]  Pulsars and magnetars
 \ben 
\item[VIII.1] Evidence that pulsars are neutron stars
\item[VIII.3] Observed properties
\item[VIII.4] Magnetohydrodynamics
\item[VIII.5] Magnetic dipole model
\item[VIII.6] Fast pulsars and the upper limit on pulsar spin
\item[VIII.7] Magnetars
\een

\item[IX.] Black holes
\ben 
\item[IX.1] Eddington-Finkelstein and Kruskal charts
\item[IX.2] Particle orbits
\item[IX.3] Model collapse
\item[IX.4] Kerr black holes
\item[IX.5] Supermassive black holes
\een

\item[X.] Compact binaries
\ben 
\item[IX.1] Roche limit and evolution of accreting binaries
\item[IX.2] White dwarf binaries
\item[IX.3] Binaries with neutron stars and black holes:\\ 
             Gravitational radiation and time to coalescence
\item[IX.3] X-ray binaries
\een

\item[XI.] Late inspiral and coalescence
\ben 
\item[XI.1] Tides, late inspiral, and the equation of state 
\item[XI.2] Short gamma-ray bursts
\item[XI.3] Tidal disruption, collision, and ejecta 
\item[XI.3] s- and r-process elements and kilonovae
\item[XI.4] Coalescence models 
\item[XI.5] Measuring the Hubble constant
\een


\een
\end{document}
